%!TEX program=xelatex

\documentclass{ctexart}
\author{TagMaker}
\title{迭代一后端总结文档}
    \begin{document}
    \maketitle
    \newpage
    \tableofcontents
    \newpage
    \section{后端开发历程简述}
        在迭代一的过程中,后端开发整体较为顺利,基本能够按计划完成。开始开发时为后端
        分配了三位同学,后来根据实际任务量调整至两位同学。
        两位同学先是分成了developer和QA两个角色,分别学习各自的技术,
        QA同学搭建好持续集成、质量检测、测试环境,developer同学
        构建基本的springboot项目,做了简单的设计
        之后两位同学分层开发,分别负责逻辑层和数据层。开发完毕后由QA同学主要
        负责测试,developer同学主要负责修复,在各种工具的帮助下很快将单元测试
        覆盖率达到了60\%以上。之后后端又对一些细节做了加强,比如动态地上传
        本地文件,这样可以在运行时加载图片。比如对代码质量进行了一定的提升,改善了
        一些编码风格。
    
    \section{后端设计中的重要决策}
        \begin{itemize}
            \item[REST] \
            
            软件使用前后端分离的restful接口进行开发,因为这样可以前后端并行开发,
            适应迭代一快速而简明的任务要求。同时REST接口将前后端的联系
            限制在数据本身,降低了耦合度,有利于之后实现功能的扩充。

            \item[分层] \

            后端的逻辑层和数据层使用分层开发的模型进行构建。因为迭代三有极大可能会使用
            数据库来增加对数据的处理能力,因此分层有利于实现这个变更。


        \end{itemize}
    \newpage
    \section{后端技术简介}
    \begin{itemize}
        \item [Springboot] \

            Springboot是后端代码构造过程中使用的主要工具。我们主要使用了Springboot的
            IoC技术来控制类与类之间的协作,用AOP技术来对异常进行了捕获和处理,
            用Springboot自带的多线程(@Schedued)进行了方便快捷的多线程编程。

        \item [Gradle] \

            gradle主要负责依赖管理、测试、生成测试报告

        \item [Jacoco] \

            jacoco和gradle配合,用来生成覆盖率报告

        \item [TestNG] \

            测试工具

        \item [Jenkins] \

            持续继承工具,每一次提交到git上之后,Jenkins会自动进行构建和测试
            并生成晴雨表。除此之外,每天的固定时间会执行一次构建。
        
        \item [Sonar] \

            进行代码质量检测,并给出修复建议

        \item [RestClient] \

            用于方便地进行Rest接口的测试,并用于生成Rest规格文档

    \end{itemize}
    \newpage
    \section{后端有待完善的地方}
        \begin{itemize}
            \item 由于复杂度较低,没有使用测试驱动开发

            \item 回归测试不够充分,测试阶段注入bug较多

            \item 新技术较多,有的利用不充分

            \item 没有自动部署

        \end{itemize}

    \section{后端迭代二中的目标}
        \begin{itemize}
            \item 充分利用自动部署工具

            \item 随着需求的复杂化,重构面向前端的restController的组织形式

            \item 增加一些统计/计算相关的模块,增加较为复杂的算法

            \item 增加单元测试力度,使测试更充分,并设立回归测试用例组进行回归测试
        \end{itemize}

    \end{document}