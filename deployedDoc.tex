%!TEX program = xelatex

\documentclass{ctexart}
        \usepackage{listings}
        \usepackage{color}
\author{TagMaker}
\title{COUNTSnju部署文档}
\begin{document}
\maketitle
\newpage
\tableofcontents
\newpage
\section{部署环境要求}
\begin{itemize}
        \item [系统要求]
        \
        

        \item\  [推荐]Linux Ubuntu 16.04

        \item\  [经过测试]macOS 10.13+

        \item[Ubuntu 环境要求]
        \

        \textcolor{red}{本文档假设Ubuntu里已具备sudo和apt,如果这两个程序出现异常,请联系你的系统管理员}
        
        \item\ JDK 1.8+

        首先,输入如下命令来检测是否已配置好Java:

        \begin{lstlisting}[language=bash]
         $java -version
                
        \end{lstlisting}

        应该得到类似这样的输出 ( 版本可以为8、9或10 )

        \begin{lstlisting}[language=bash]
         java version "8"
                
        \end{lstlisting}
        \textcolor{red}{如果你没有得到上述输出,请执行如下命令}
        \begin{lstlisting}[language=bash]
         $sudo apt-get install openjdk-8-jdk
                       
        \end{lstlisting}
        然后输入系统的登录密码,并在接下来的提示中输入 " \emph{y} "

        \item\ vim

        输入如下命令来检测是否已安装vim:

        \begin{lstlisting}[language=bash]
         $vim -version
                
        \end{lstlisting}

        应该得到类似这样的输出

        \begin{quote}[language=bash]
                VIM - Vi IMproved 7.4 (2013 Aug 10, compiled Nov 24 2016 16:44:48)
                Garbage after option argument: "-version"
                More info with: "vim -h"
                
                
        \end{quote}

        \textcolor{red}{如果你没有得到上述输出,请执行如下命令}
        \begin{lstlisting}[language=bash]
         $sudo apt-get install vim
                       
        \end{lstlisting}
        然后输入系统的登录密码,并在接下来的提示中输入 " \emph{y} "


        \item\ 互联网接入



        
\end{itemize}



\end{document}
